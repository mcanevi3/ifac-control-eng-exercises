\documentclass[a4paper,12pt,addpoints]{article} 
\usepackage{amsmath,amssymb,amsfonts}
\usepackage{graphicx}

\begin{document}

Given a repository of questions mapped to intended learning outcomes(ILOs) the generation of an exam by choosing a set of questions out of the repository is a crucial task for educators.
An exam generation based on Bloom's taxonomy from a repository of questions randomly is mentioned in \cite{amria2018framework}. 

An iterative exam generation based on a database of questions prepared using \verb|MATLAB| and generated using \verb|LaTeX| is given in \cite{stotsky2024automatic}.

A rather more control design approach is provided in \cite{chow2024personalized}, where adaptive cruise control algorithm is adopted to adjust the accomplishment of students to the study material goal in terms of ILOs from a repository of questions instead of choosing questions from the repository randomly. A survey is applied to the student beforehand and the instructor is assisted via the so called \verb|iPRACTISE| system. 

The use of personalized practice quizzes utilizing the \verb|STACK| framework has been documented for advanced mathematics relevant to control engineering, focusing on topics such as the Solution of the Euler-Lagrange differential equation in \cite{erskine2018developing}. The exercises are staged into questions to provide feedback on early parts of the calculation before students proceed to the final, more complex steps.

A \verb|MATLAB| Cody Coursework approach is applied in detail in \cite{hill2018automated}. The approach is based on students coding rather than providing a pen and paper solution or choosing from multiple-choice answers. An extensive examination of designing assignments in the control engineering course context is provided.

An e-learning framework using \verb|MATLAB| apps is applied at the University of Stuttgart in the Introduction to Automatic Control course \cite{romer2019facilitating}, where it is observed that the interactivity and gamification via the apps improved the student experience.


\bibliographystyle{ifaconf}
\bibliography{library}
\end{document}