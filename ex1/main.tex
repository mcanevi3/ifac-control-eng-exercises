\documentclass[a4paper,12pt,addpoints]{exam} 
\usepackage{refcount}
\usepackage{amsmath,amssymb}
\usepackage{graphicx}
\usepackage{amsmath}
\usepackage{amsfonts}
\usepackage{xcolor}
\usepackage[shortlabels]{enumitem}
\usepackage{tikz}
\usetikzlibrary{shapes,arrows}

\tikzstyle{block} = [draw, fill=blue!20, rectangle, 
    minimum height=3em, minimum width=6em]
\tikzstyle{sum} = [draw, fill=blue!20, circle, node distance=1cm]
\tikzstyle{input} = [coordinate]
\tikzstyle{output} = [coordinate]
\tikzstyle{pinstyle} = [pin edge={to-,thin,black}]
\usetikzlibrary{angles, arrows.meta,
                quotes}
\usetikzlibrary{fit,calc,positioning,decorations.pathreplacing,matrix}                
\usepackage{pgf} 
\usepackage{pgfmath}
\usepackage{pgfplots}
\usepackage{bodegraph}
%\pgfplotsset{compat=1.3}
\pgfplotsset{compat=1.17}
%\usepgflibrary{fpu}
%\pgfkeys{pgf/fpu}

\def\x{0}
\def\y{1}

\usepackage[randomize,nokeeplast,overload]{exam-randomizechoices}
\pgfmathparse{10*\x+\y}
\setrandomizerseed{\pgfmathresult}
\keylistquestionname{Question}
\keylistkeyname{Answer}

\begin{document}

x:$\x$
y:$\y$

\begin{questions}
\question[5] A complex number is given as $\pgfmathparse{(\x+2)}\pgfmathprintnumber{\pgfmathresult}+\pgfmathparse{(\y+2)}\pgfmathprintnumber{\pgfmathresult}j$ where $j=\sqrt{-1}$.
Which of the following is the absolute value of the given complex number?
\begin{randomizechoices}
\choice \pgfmathparse{sqrt((\x+2)+(\y+2))}\pgfmathprintnumber{\pgfmathresult}
\choice \pgfmathparse{sqrt((\x+2)+(\y+2)+1)}\pgfmathprintnumber{\pgfmathresult}
\choice \pgfmathparse{(\x+2)*(\x+2)+(\y+2)*(\y+2)}\pgfmathprintnumber{\pgfmathresult}
\choice \pgfmathparse{(\x+3)*(\x+3)+(\y+3)*(\y+3)}\pgfmathprintnumber{\pgfmathresult}
\CorrectChoice \pgfmathparse{sqrt((\x+2)*(\x+2)+(\y+2)*(\y+2))}\pgfmathprintnumber{\pgfmathresult}
\end{randomizechoices}
\end{questions}

\printanswers

\ifprintanswers
\printkeytable
\writekeylist
\fi
\end{document}